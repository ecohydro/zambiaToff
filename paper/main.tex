%%%% added from https://groups.google.com/forum/#!topic/latexusersgroup/CDlEjgNnF80

%\documentclass[onecolumn]{rsauthor}    
\documentclass[a4paper]{article}
\usepackage[affil-it]{authblk}    


\usepackage{graphicx}
\usepackage{tabularx}
\usepackage{epstopdf}
\usepackage{amsmath}
\usepackage{amssymb}
\usepackage{amsfonts}
\usepackage{amsthm}
\usepackage{endfloat}
\usepackage[numbers,sort,compress]{natbib}
%\bibpunct{[}{]}{,}{n}{,}{,}  % https://xianblog.wordpress.com/tag/natbib/ (allows natbib with PNAS)
\usepackage{endnotes}
\usepackage{setspace}
\usepackage{verbatim}
\usepackage[left=2.5cm, right=2.5cm, bottom=2cm, top=2cm]{geometry}
\usepackage{times}
\usepackage{helvet}
\usepackage{courier}
%\usepackage{mathtime}
\usepackage{bm}
\usepackage{url}
%\usepackage{babel}
\usepackage{dcolumn}
\usepackage{multirow}
%%%%%%

\usepackage[
  breaklinks=true,
  colorlinks=true,
  linkcolor=blue,anchorcolor=blue,
  citecolor=blue,filecolor=blue,
  menucolor=blue,pagecolor=blue,
  urlcolor=blue]{hyperref}

% Let's add todonotes and comments:
%\usepackage[draft]{todonotes}
%
%% Select what to do with command \comment:  
%% \newcommand{\comment}[1]{}  %comment not showed
%\newcommand{\comment}[1]
%{{[\bfseries \color{blue} #1]}} %comment showed

\usepackage{lineno}
\usepackage{float}
\usepackage[anythingbreaks]{breakurl}

\title{Reconciling agriculture, carbon, and biodiversity in a savanna transformation frontier}
\author[1,2]{Estes, L.D.}
\author[2]{Searchinger, T.}
\author[1]{Spiegel, M.}
\author[1]{Tian, D.}
\author[3]{Sichinga, S.}
\author[3]{Mwale, M.}
\author[4]{Kehoe, L.}
\author[4]{Kuemmerle, T.}
\author[1]{Berven, A.}
\author[5]{Chaney, N.}
\author[1]{Sheffield, J.}
\author[1]{Wood, E.F.}
\author[1]{Caylor, K.K.}
\affil[1]{Civil and Environmental Engineering, Princeton University, Princeton, NJ, 08544 USA}
\affil[2]{Woodrow Wilson School, Princeton University, Princeton, NJ, 08544 USA}
\affil[3]{Zambia Agricultural Research Institute, Mt. Makhulu Research Station, Chilanga, Zambia}
\affil[4]{Geography Department, Humboldt University, 10099 Berlin, Germany}
\affil[5]{Program in Atmospheric and Ocean Sciences, Princeton University, Princeton, NJ USA}
\date{}

\begin{document}
\maketitle
%\begin{frontmatter}

\begin{abstract}
Rapidly rising populations and likely increases in incomes in sub-Saharan Africa make tens of millions of hectares of cropland expansion nearly inevitable, even with large increases in crop yields. Much of that expansion is likely to occur in higher rainfall savannas, with substantial costs to biodiversity and carbon storage. Zambia presents an acute example of this challenge, with an expected tripling of population by 2050, good potential to expand maize and soybean production, and large areas of relatively undisturbed miombo woodland and associated habitat types of high biodiversity value. Here we present a new model designed to explore the potential for targeting agricultural expansion in ways that achieve quantitatively optimal trade-offs between competing economic and environmental objectives: total converted land area (the reciprocal of potential yield); carbon loss, biodiversity loss, and transportation costs. To allow different interests to find potential compromises, users can apply varying weights to examine the effects of their subjective preferences on the spatial allocation of new croplands and its costs. We find that small compromises from the objective to convert the highest yielding areas permit large savings in transportation costs, and the carbon and biodiversity impacts resulting from savanna conversion. For example, transferring just 30\% of weight from a yield maximizing objective equally between carbon and biodiversity protection objectives would increase total cropland area by just 2.7\%, but result in avoided costs of 27\%-47\% for carbon, biodiversity, and transportation. Compromise solutions tend to focus agricultural expansion along existing transportation corridors and in already disturbed areas. Used appropriately, this type of model could help countries find agricultural expansion alternatives and related infrastructure and land use policies that help achieve production targets while helping to conserve Africa's rapidly transforming savannas.


%Rapidly rising populations and likely increases in incomes in sub-Saharan Africa make tens of millions of hectares of cropland expansion nearly inevitable even with large increases in crop yields.  Much of that expansion is likely to occur in higher rainfall wetter savannas, at substantial costs to biodiversity and carbon storage. Zambia presents an acute example of this challenge with an expected tripling of population by 2050, good agronomic potential to expand maize and soybean production, and large swaths of relatively undisturbed Miombo woodland and associated habitat types of high value to mammals and birds.  Here we present a new model designed to explore the potential to target agricultural expansion in ways that achieve reasonable trade-offs among different economic and environmental objectives.  To allow different interests to find potential compromises, the model allows users to apply different weights to different objectives and examine how they alter outcomes.  We find that modest compromises in yield targets permit large savings in transportation costs, and carbon and biodiversity impacts from savanna conversion. For example, transferring just 30\% of weight from a yield maximizing objective equally to carbon and biodiversity would increase total cropland area by just 2.7\%, but result in avoided cost of 27\%-47\% for the carbon, biodiversity, and transportation cost objectives. Compromise solutions tend to focus agricultural expansion along existing transportation corridors and in already disturbed areas.  Used appropriately, this type of model could offer a practical means by which countries could explore agricultural expansion alternatives and target transportation investments and land use policies to achieve production targets while helping to conserve the values of Africa's savannas.     
\end{abstract}
%% Text of abstract

%\begin{keyword}
%keywords
%\end{keyword}

%\end{frontmatter}


\linenumbers

%% main text
\section*{\large Introduction}
Meeting growing food demands while minimizing ecological losses presents one of the 21st Century's major challenges.  Many global sustainability studies and climate mitigation pathways call for eliminating or rapidly phasing out emissions from land use change \cite{searchinger_high_2015,field_summary_2014}.  Unfortunately, these ambitions do not appear realistically attainable in sub-Saharan Africa, due to a projected doubling of population by 2050 combined with rapidly growing  economies, which will combine to treble food demand \citep{tilman_global_2011,searchinger_high_2015}. For example, even with the substantial yield growth (3-5\% per year) projected by the UN Food and Agriculture Organization (FAO), the region would likely need to add more than 100 million hectares of cropland by 2050 \citep{searchinger_high_2015}. Much and probably the great bulk of that new cropland is likely to occur in the ~500 million hectares of savannas and shrublands of the region that receive sufficient rainfall to permit crop production \citep{searchinger_high_2015}. 

Savannas sometimes receive limited respect in global land use assessments and have been a major focus of agricultural expansion, not only in Africa but in places such as the Cerrado of Brazil and northern Thailand \citep{morris_awakening_2009}.  Their carbon contents and biodiversity values may be completely ignored on the grounds that they are not forests \citep{searchinger_high_2015}. Yet, the wetter woodland-savannas and shrublands of Africa have a similar average richness, although not density, for birds and mammals as the world's wetter tropical forests, and 75\% of these habitats have higher overall vertebrate diversity than 75\% of the rest of the world \citep{searchinger_high_2015}.  Their carbon contents are also substantial, particularly relative to their likely yields, so that their use for food crops is unlikely to result in fewer carbon emissions per ton of crop \citep{searchinger_high_2015}.  Conversion of over 100 million hectares of these lands would release tens of millions of tons of carbon \citep{searchinger_high_2015}.  

The high likelihood of some cropland expansion in sub-Saharan Africa raises the prospect of whether this expansion could be located in ways that meet agricultural production goals, but for substantially lower loss of carbon and biodiversity than conventional agricultural development pathways.  Analyzing this question must be done at national or sub-national scales, because that is where key decisions are made, and accurate results require data that are high in both resolution and quality.

In this paper we look at Zambia, which is a bellwether for this land use challenge in sub-Saharan Africa. On the one hand, Zambia has great need to boost its food production. Sixty percent of Zambia's population is rural {\citep{the_world_bank_world_2016}, per capita calorie food availability is just 2100 kcal \citep{fao_faostat_2016}, and child stunting rates are 40\% \citep{haddad_global_2015}.  According to the medium estimate of the United Nations, Zambia's population is projected to grow 265\% from 16.2 million in 2015 to 43 million in 2050 \citep{united_nations_world_2015}, which will combine with rapid economic growth \citep[$>$6\%,][]{the_world_bank_world_2016} to substantially increase per capita calorie demands \citep{tilman_global_2011}. Zambia has also been expanding its agricultural exports, and has agronomic and situational characteristics that put it in a strong position to continue to do so to other parts of the region. 
 
At the same time, two thirds of Zambia's 738,400 km$^2$ land area is dominated by woodland savannas, which have at sufficiently dense tree cover to be classified as forest under either of the definitions ($>$10 or $>$30\% tree cover) used by the United Nations Framework Convention on Climate Change \citep{sexton_conservation_2015}.  These ecosystems store substantial amounts of carbon, with 50\% of Zambia having a carbon density of at least 130 t ha$^{-1}$ (vegetative carbon and soil carbon in the top 1 m), while an additional 40\% exceeds 80 t ha$^{-1}$. Meanwhile, Zambia's deforestation rate of 2,500--3,000 km$^2$ year$^{-1}$ is one of the highest in the world \citep{vinya_preliminary_2011}.
 
The biodiversity of these woody savannas is also high. Most of Zambia is covered by what the World Wildlife Fund has categorized as the Central Zambian Miombo Woodland ecoregion \citep{olson_terrestrial_2001,hogan_central_2004}, which contains some 3800 plant species, making it the 17th richest ecoregion in the world in floral diversity (out of 867) \citep{kier_global_2005}. Vegetation is characterized by a mix of tall, frequently evergreen, trees (15-20 m), broadleaf shrubs, and grasses, and is heavily interspersed with wetlands \citep{hogan_central_2004}.  The large expanses of relatively intact habitat give this woodland the third highest richness for mammal species of the worlds ecoregions  \citep{olson_terrestrial_2001,fund_ecoregions_2011, estes_zambezian_????}.  It supports rare species including the  black rhino,  major predators, and highly migratory, wide-ranging species, such as elephants \citep{hogan_central_2004}.  Zambia overall also has a high bird diversity, with 753 recorded species \citep{dowsett_birds_2008}. Although few bird and mammal species are endemic, reptile and amphibian endemism is high, with 19 endemic reptile species and 13 endemic amphibians \citep{fund_ecoregions_2011}.  
 
There is a long history of using models to achieve conservation objectives in the face of cost and other constraints \cite[e.g.][]{naidoo_integrating_2006,margules_systematic_2000}. MARXAN, perhaps the most widely used of these, is a decision-support program that is designed to efficiently select conservation area networks \citep{ball_marxan_2009,watts_marxan_2009}. Within the more general land use planning field, an even broader range of models exists for finding optimal spatial configurations that satisfy multiple land use objectives (see Cao et al. \citep{cao_spatial_2011} for an example and review of alternatives). In the present case, the need is not to select protected areas but to identify areas for cropland expansion that meet the objectives of agricultural development and environmental conservation. Koh and Ghazoul \citep{koh_spatially_2010} developed one model for examining trade-offs among oil palm development, rice production, and carbon and biodiversity production in Indonesia, and a related approach was used to identify sustainable agricultural intensification solutions in southern Tanzania \citep{nijbroek_regional_2016}. In both examples, the model assigned equal weight to each land use objective in the compromise scenarios that were analyzed. However, one of the great challenges for land use planning is that different objectives can be difficult to weigh against each other \emph{ex ante}, because the values are often abstract, making them harder to evaluate without first viewing the consequential trade-offs \citep{cao_spatial_2011,nijbroek_regional_2016}. Governments and other stakeholders also tend to have varying, and often conflicting, priorities, and their willingness to compromise may depend on the degree to which their priorities are impacted.

To overcome these problems, we developed a new, publicly available model that allows varying levels of compromise between competing agricultural and environmental interests to be evaluated while still meeting agricultural production targets. The model adopts different methods than the two aforementioned models \citep{koh_spatially_2010,nijbroek_regional_2016} to estimate agricultural potential and conservation objectives, and can allocate land for multiple crops simultaneously. More fundamentally, our model is based on linear programming, which allows users to place varying weights on the different land use objectives \citep{cao_spatial_2011}. This feature allows different stakeholders to work together to choose the most appealing combination of weights based on their land use preferences, and their own assessments of what are and are not acceptable costs within those objectives. 

We apply this model to Zambia so that we can explore its strengths and limitations as a decision-making tool, and also to answer a basic but important question: is it possible to achieve a reasonable balance of different interests by reducing the carbon and biodiversity costs of expansion while still focusing on land use areas with high agricultural potential?  Answering this question is becoming increasingly important as savannas and other grassy biomes receive increasing agricultural development pressure, and may become increasingly targeted as an indirect result of the laudable efforts to conserve tropical forests \citep{searchinger_high_2015}. 

\section*{\large Materials and Methods}

The model we developed seeks to find optimal land use configurations that satisfy the production targets for multiple crops while minimizing four costs: the total land area required, transportation costs, carbon released from land conversion, and impacts on biodiversity. Although total land area is treated as a cost, minimizing this cost identifies lands with the highest likely yields. By minimizing total land required as well as transportation costs, the model therefore finds areas that reflect two important indicators of agricultural potential. By minimizing carbon and biodiversity costs, the model also finds areas that minimize two important dimensions of environmental impact.  The model chooses final areas depending on the weights that a user assigns to each of these four objectives.   

To enable this functionality, and to identify optimal solutions, the model is structured to evaluate each objective in terms of production efficiency, or the ratio of each cost per hectare to crop production per hectare (yield) that results from converting land to agriculture.  For land area, cost is measured as the number of km$^2$ required to meet the crop production target. For the other objectives, the costs indicators are transportation time, carbon loss, or biodiversity impacts.  

We focus our model initially on two crops, maize and soybeans, because of their existing and potential significance for Zambia's agricultural sector, and their likely dominance of in the future growth of agriculture in Africa's broader savanna regions \citep{searchinger_high_2015}.  Maize is the largest crop by area and consumption in Zambia \citep{fao_faostat_2016}, while soybeans are expanding rapidly as a commodity crop to meet regional and global demands for animal feed and direct human consumption \citep{gasparri_emerging_2015}. Because the model is designed to meet production targets for multiple crops, we apply algorithms for allocating each crop to its highest value cells, given the particular combination of objectives being assessed.   

Further details on model structure and all model input calculations can be found in the online electronic supplement.  

\subsection*{Model inputs} 

Our model focuses on the costs of cropland expansion into areas that are currently not farmed but could be.  To identify such areas, we calculated the proportion of currently cropped and settled grid cells within each 1 km$^2$ grid cell.  We excluded the proportion of cells that are either protected or with slope greater than 20\% slope, which are both generally considered unsuitable for farming and strong actual predictors of cropland presence or absence \citep{estes_using_2014}.  We estimated the costs of converting each cell as described in the following sections.   

\subsubsection*{\emph{Likely yields}}

We spatially estimated likely yields for maize and soybeans using a three-step approach. In the first step, we used the decision support system for agrotechnology transfer \cite[DSSAT, ][]{jones_modelling_2003} to simulate maize and soybean yields at the sites of 40 weather stations distributed across Zambia. They are the points where a 31-year, gridded daily meteorological dataset has received the greatest amount of bias-correction and infilling by weather observations \citep{sheffield_development_2006,chaney_spatial_2014, estes_changing_2014}. We used soil profiles corresponding to those locations drawn from the WISE v1.1 gridded soil profile database \citep{harvestchoice_converting_2010}, and simulated for each location maize yields over all 31 years under commercial input practices for three-four cultivars representative of short and medium season length according to the Zambian crop calendar \citep{grassini_how_2015}.  

In the second step, we used an empirical model to map the DSSAT-modeled yields onto a 1 km grid, taking advantage of a finer resolution soil map that provides a common frame for estimating both yield potential and soil carbon stocks (see below), but has insufficient detail in several parameters (particularly effective rooting depth) required by DSSAT to provide a sound empirical basis for running the model directly.  We used a generalized additive model \citep{wood_mgcv_2001} to predict the values of the DSSAT-simulated yields, using best-fitting subsets of the following weather and soil variables: growing season precipitation; growing degree days; mean growing season shortwave solar radiation; percent organic carbon; pH; percent clay content. The soil predictors represented mean values in the top 1 m of soil, and were extracted from a new 1 km global database of soil properties \cite{hengl_soilgrids1km_2014}.  This two step approach allowed us to more soundly capture crop responses to Zambia's climatic variability and potential management practices, and to base potential yields on the same dataset used to estimate carbon costs (see below). 

The first two steps resulted in a map of potential yield under high inputs and good management, and assuming no losses due to other limitations such as diseases. These yield values therefore overestimate what farmers are truly likely to achieve. To reflect more realistic estimates of future yields, we therefore rescaled the yield maps to match FAO-projected Zambian yields in 2050 (4.4 t ha$^{-1}$ for maize and 3.6 t ha$^{-1}$ for soybean).  We used the FAO-projections for sub-Saharan Africa as a whole to derive an annual growth factor and applied that factor to the average 2009-2014 yields for each crop in Zambia. Overall, the distribution of yields derived from our crop modeling is most important to targeting land use, while this choice of average yield determines the total quantity of land needed to meet production targets. 

\subsubsection*{\emph{Transportation Costs}}

As a second measure related to agricultural potential, we estimated the travel time to the nearest major market town. Travel time serves here as a proxy for differences in output prices and production costs, such as fertilizer and transport costs \cite{stifel_isolation_2008}, and, via the density of road networks, is strongly correlated with the suitability of land for farming \citep{laurance_estimating_2015,estes_using_2014}. For this analysis, we defined market towns as the administrative capitals of Zambia's districts, as well as any town outside of Zambia having a population of $\geq$10,000 within a region bounded by 19.5$^{\circ}$ to 34.5$^{\circ}$ longitude and -19.5$^{\circ}$ to -8$^{\circ}$ latitude. These neighboring towns were included because they may be more easily accessible to Zambians living in border regions than the nearest district capital. 

We calculated travel time in hours using cost-distance analysis drawing on several spatial datasets. Vector data for Zambian roads obtained from the Government of Zambia (GOZ) and for the broader region from OpenStreetMap\footnote{http://openstreetmap.org} were merged to create a single dataset of regional roads that distinguished between trunk, primary, secondary, and tertiary roads. For market towns, we obtained district capital towns locations from the GOZ and for towns outside Zambia we used data from the Gridded Population of the World Version 3 \citep{ciesin_gridded_2005} database. For waterbodies, we used the USGS Hydrosheds database \citep{lehner_new_2008} to indicate the position and size of rivers, based on their contributing area, and used a lakes vector provided by the GOZ. The roads, rivers, and lakes data, as well as Zambia's border, were converted to 1 km grids wherein each feature was assigned a travel time value based on how long it would take to traverse 1 km. The resulting grids were merged into a single ``friction'' surface, which was used to compute the number of travel hours along the fastest route to the nearest market town (see online electronic supplement for further details).  

Total travel costs for any land use scenario equal the total time needed to transport the tonnage of crops produced in each converted cell to the nearest town, We divided these costs by 20 to produce "truck hours" on the assumption of a 20-ton truck. Doing so produces a more intuitive unit, but does not alter mapping outcomes.

\subsubsection*{\emph{Carbon}} 

To estimate potential carbon costs from cropland conversions, we developed maps of vegetative and soil carbon stocks for Zambia.  Vegetative carbon was calculated from a recent map of above-ground biomass developed by Baccini et al. \citep{baccini_estimated_2012}. We estimated below-ground biomass using savanna and miombo woodland-specific root-shoot ratios \cite{mokany_critical_2006, mugasha_allometric_2013}, and then converted the total biomass to carbon using a ratio of 0.47 \citep{ipcc_land_1996}. The same soil carbon dataset used for crop yield prediction, together with an accompanying map of bulk density, was used to calculate soil carbon stocks in the top 1 m. We calculated the potential carbon loss due to agricultural conversion was calculated as 100\% of vegetative carbon and 25\% of soil carbon \cite{searchinger_high_2015}. 
 
\subsubsection*{\emph{Biodiversity}}

There are many ways of evaluating biodiversity value and potential impacts, including species richness of different taxa in absolute terms or based on level of threat or endemism \citep[e.g.][]{searchinger_high_2015, gasparri_emerging_2015}. Other possibilities include factors designed to address the significance of contiguity or the impact of different types of development on diversity \citep[e.g.][]{newbold_global_2015,koh_spatially_2010}.  The actual measures selected should reflect both conservation priorities and the quality of available data.  For a fine-scaled, country-level analysis such as this one, the species range map data that form the basis for many broader impact assessments \citep[e.g.][]{kehoe_global_2015} have an effective resolution that is too coarse. We therefore developed a composite measure of biodiversity value based on several datasets. 

First, we used the potential vegetation maps of East Africa \citep[][]{van_breugel_potential_2011}, a 2010 landcover dataset for Zambia\footnote{downloaded from http://apps.rcmrd.org/landcoverviewer/}, and a shapefile of protected areas to calculate how much of Zambia's 20 unique vegetation types remained after conversion to croplands and settlements as of 2010, and what proportion of that remainder fell within national parks. We used these values to calculate an index of rarity and threat. We calculated rarity as the proportion of Zambia occupied by each vegetation type's remaining area, which we then log-transformed to account for the highly skewed distribution. For threat, we calculated the proportion of each vegetation type falling within protected areas, and then multiplied the threat and rarity values to create the index \citep{van_breugel_environmental_2015}. We then used the remaining vegetation map to calculate a second measure, the proportion of untransformed vegetation within a 11x11 km neighborhood centered on each 1 km$^2$. By adding this measure to the first index and normalizing, we created a biodiversity score with a range of 0-1 per ha, which gives equal weight to a habitat based on a) how rare or threatened and b) how undisturbed it is.  Summing the biodiversity score for each converted grid cell generates a total cost of biodiversity impacts for each scenario.  

As a final modification, we masked out the areas of existing national parks and game management areas to respect the national judgment that they should not be used for expanding farmland.  We retained a third protected category, forest reserves, as these tend to be far less stringently protected, having already lost 8.3\% of their area to cropland or settlements, but assigned these areas a high biodiversity score (0.927, the inverse of their converted proportion). 

\subsubsection*{\emph{Potential farmland and production targets}}

Likely future demand for maize and soybeans produced in Zambia is difficult to estimate, and our model can easily be run with different production targets.  For this paper, we adopted targets of a four-fold increase for maize for 2050 compared to 2009-2014 average, and a 10-fold increase for soybeans.  These increases reflect the rate of growth in Zambia from 2000-2014, and a mean production trend for soybean growth in southern Africa estimated by Gasparri et al \citep{gasparri_emerging_2015}, with small additional adjustments to account for the fact that Zambia has one of the highest agricultural potentials in sub-Saharan Africa and has high potential to expand exports.  

\subsubsection*{\emph{Yields on current land}}

The amount of land conversion to meet a production target will depend on the growth of yields on existing croplands, which are well short of their current potentials \citep{waddington_getting_2010}.  For the principal scenarios we present in this paper, we assume that yields on current cropland will achieve the average likely yields for Zambia's \emph{potential} farmland.  However, to examine the possible significance of yield gains on existing cropland, we also examined scenarios in which the yield gains on existing cropland 1) only manage to close 50\% of the gap between current and our projected future yields, and 2) exceed the gap by 50\% (i.e. achieve 150\% of the expected yield gain). 

\subsection*{Normalization and Weighting} 

Weighting different objectives to some extent depends purely on preferences, but it is important that these preferences be expressed in a numerical form that reasonably reflects them so that a quantitative solution can be computed. To achieve this result, we first express the different objectives as efficiencies, e.g. carbon loss per ton of crop yield, and then normalize the units of each objective based on the range of its scores. Weights representing the percentage of importance are placed on each objective, expressed as a decimal between 0 and 1, with values closer to 1 indicating stronger preferences, and with all weights summing to 1. The final score is the sum of the resulting products for each objective for each grid cell, which is used to rank each grid cell according to its ability to meet the production target most for the lowest total cost, given those weights. The model then selects cells in descending rank order until the cumulative production of converted cells reaches each crop's corresponding production target.

Although this weighting system treats choices as relative preferences, our model calculates the actual costs of each objective in absolute terms:  total area converted, total transportation costs from new croplands, total carbon released from land conversions, and total impacts on biodiversity. After reviewing these costs, model users can then adjust the preferences to reflect desirable or politically acceptable trade-offs.

\section*{\large Results}

\subsection*{Significance of yield gains on existing cropland}

In our main scenario, we assumed that the gap between current and projected future yields would be fully closed on existing cropland, but we also ran the model for cases when only 50\% or 150\% of the gap was closed (with equal weights across all objectives). In the former case, the total land area required to meet the target increased by 18\% (14,634 to 17,202 km$^2$), transport costs increased 22\% (5.52 x 10$^{-5}$ to 6.61 x 10$^{-5}$ truck hours), carbon losses increased by 19\% (597 to 711 Mt), and the total biodiversity index of converted areas increased by 20\% (5.53 x 10$^{-5}$ to 6.61 x 10$^{-5}$).   In the 150\% gap closure scenario, the impacts were reversed, with impacts declining by almost exactly the same percentage. 

\subsection*{Overlap and adjacency from optimal individual solutions}

To identify the minimum possible costs for each individual objective, and the degree to which their areas of new cropland correspond, we examined simulations giving 100\% of the weight to each of the four different objectives. Figure 1 maps the results, and shows the costs for each objective in each simulation, as well as the extent of overlap among objectives. 

The scenario designed solely to maximize yield (100\% weight) converts land with the most promising yield potential for farmers, which should have economic advantages.  These areas have little direct overlap with the areas picked by scenarios that minimize carbon costs, biodiversity impacts or travel time (Figure 1).  They also have little adjacency, as measured by the average distance between each pixel of one map and its nearest neighbor in the other (Figure 1). 
 
The areas selected solely for minimizing carbon, biodiversity, or transportation costs also have little direct overlap, but the map reveals that they have high adjacency, confirmed by direct estimates of distance (Figure 1). All three criteria tend to identify lands along existing road networks and population and agricultural centers, with the carbon minimizing objective converting a larger area of land in the western half of the country. The reasons for this correlation are probably that carbon stocks and biodiversity values are lowest in areas of high human densities, which have correspondingly high levels of wood harvesting for charcoal and fuelwood, and higher levels of habitat loss and fragmentation leading to lower biodiversity scores.  

\begin{figure}[!ht]
  \begin{center}
       \makebox[\textwidth][c]{\includegraphics[width=1.2\textwidth]{figures/figure1.pdf}}
    \caption{Areas converted to new maize and soybean croplands and their associated costs when each individual objective (yield maximization, and carbon cost, biodiversity cost, and transport time minimization) is given 100\% weight. Cropland converted under each scenario is shown in the maps, colour-coded according to which objective receives 100\% weight. Large bar charts, colour-coded to the maps, display the absolute value of costs resulting from each scenario. Small bar charts show the percentage of overlap (top) and adjacency (bottom; nearest neighbor distance) between the map directly above it, and each of the other three maps.}
    \label{default}
  \end{center}
\end{figure}
 
\subsection*{Potential to harmonize different interests}

Although the ideal solutions to maximize each objective do not lead to much direct overlap, areas selected by less ideal but still good solutions for each objective may still overlap.  The reasons start with the different potential ranges of costs among the scenarios that maximize each objective.

While the best case scenario for maximizing yield would convert 13,668 km$^2$, the worst case scenario for yields among our single-purpose optimization scenarios would convert 15,302 km$^2$ (the carbon-optimizing scenario). This range in converted areas is only 12\%. This difference is far lower than the range in yields across the country's potential cropland (3.7-6.1 t/ha/y ha$^{-1}$ for maize and 0.8 to 5.1 t/ha/y ha$^{-1}$ for soybean), but yield plays such a significant role for each objective (because it is always in the denominator) that the model selects land with better yield potential even when the yield maximization priority is given no weight.  By contrast, the differences in travel times among single-objective scenarios range from a total of 3.4 x 10$^{-5}$ to 32.3 x 10$^{-5}$ truck hours, a difference of 872\%.  Carbon costs differ by 156\% (462 to 1184 Mt), and biodiversity by 68\% (4.96 x 10$^{-5}$ to 8.33 x 10$^{-5}$).  These results indicate that there is a strong potential for reducing transportation, carbon and biodiversity costs with only a small increase in additional land area needed to meet production targets. In other words, these cost savings can be achieved for relatively little sacrifice of potential yield, which is inversely related to converted area.   

This potential is illustrated in Figure 2, which plots the ``efficiency frontier'' \citep{polasky_where_2008} resulting from the range of possible tradeoffs between the yield maximization and carbon protection objectives. Each point represents the most efficient solution, in terms of lowest possible carbon loss and land area converted, for meeting the crop production targets under the given combination of weights. Color-coded maps indicate the associated cropland conversions for 5 of the 21 weighting combinations. Adding just 5\% weight to carbon protection (orange scenario) results in 31\% less carbon loss but $<$1\% additional converted area relative to the pure yield maximization objective (red scenario), without significantly altering the location of the new cropland (adjacency = 7.4 km). Transferring 25\% weight to carbon (yellow scenario) substantially shifts the conversion map (adjacency to pure yield maximization map = 65 km), but results in 54\% less carbon loss. These savings nearly equal the 61\% reduction in carbon loss achieved under the pure carbon protection scenario (dark blue).

\begin{figure}[!ht]
  \begin{center}
       \makebox[\textwidth][c]{\includegraphics[width=0.8\textwidth]{figures/figure2.pdf}}
    %\includegraphics[scale=0.5]{fig2.eps}
    \caption{The costs of the optimal (most efficient) tradeoffs between the yield maximization and carbon protection objectives under varying  combinations of weights (varied in 5\% increments). Costs are measured in total carbon loss and converted area. The conversion areas resulting from a subset of the weight combinations are illustrated in the accompanying maps, and are color-coded to their corresponding costs on the scatter plot, which represents the efficiency frontier. The weightings (expressed as proportions) associated with this subset of scenarios are also provided: Y=yield maximization objective; C=carbon protection objective.}
    \label{default}
  \end{center}
\end{figure}

These last examples illustrate that tradeoffs can be usefully assessed in terms of avoided cost, or the percent difference between the cost paid under a given weight combination and the maximum cost generated by all possible weighting scenarios. By expressing avoided cost as a percentage, we can directly compare the impacts of tradeoffs between more than two objectives. Figure 3 shows how much carbon, biodiversity, and transport cost can be avoided by transferring increasing amounts of weight away from the yield maximization objective. Sequentially adding weights in 5\% increments to both the biodiversity and carbon objectives results in little cost in land area, but rapid, large reductions in both biodiversity and carbon costs. It also results in substantial avoided transport costs, even without assigning any weight to this objective, because of correlations between savings in transport, carbon, and biodiversity. For example, adding just 15\% weight each to carbon and biodiversity increases the area of new lands (i.e. reduces the average yield) by just 2.7\% (avoided area costs drop to 8\% from 10.7\%), but avoids potential costs equal to 27\% for biodiversity, 47\% for carbon, and 44\% for transportation. For comparison, in the best case  scenario for each objective, 61\%, 41\%, and 90\% of carbon, biodiversity, and transportation cost would be avoided.   

\begin{figure}[!ht]
  \begin{center}
       \makebox[\textwidth][c]{\includegraphics[width=0.8\textwidth]{figures/figure3.pdf}}
    %\includegraphics[scale=0.5]{fig2.eps}
    \caption{The impact of transferring weights from the yield maximization objective to the carbon and biodiversity objectives in 5\% increments, measured in terms of the percent cost that is avoided in each objective relative to the highest cost paid across all weight combinations. In these results, no weight was assigned to the transport cost objective.}
    \label{default}
  \end{center}
\end{figure}

Figure 4 shows the potential for compromise in which a hypothetical land user only interested in a single objective is willing to ``pay'' 5\% more costs than the lowest possible costs for that objective. For example, the yield maximizing user would be willing to pay for an extra 683 km$^2$ of new cropland, while the person interested in minimizing carbon costs would pay an additional 231 Mt of carbon.  Lines in the bar charts show the minimum and maximum costs across all possible weighting scenarios.  Each of the four different ``compromise'' scenarios, represented by a different colour, permits solutions that allow the other scenarios to save substantially relative to maximum potential costs. The carbon, biodiversity, and transport cost objectives in all but two cases avoid more than 50\% of their maximum potential costs when any of the other objectives is willing to pay 5\% more cost (Figure 4).  

The location of the conversions associated with each of these compromises show substantial convergence along roads and population and agricultural centers. Overlap between the resulting cropland allocations exceeds 30\% between the yield and biodiversity and cost compromises, and 25\% between the biodiversity and carbon compromises. The carbon and cost compromises overlap least ($<$10\%), followed by carbon and yield ($\sim$15\%), but each pair of maps has high adjacency, with all compromise allocations having an average nearest neighbor distance of $<$20km.  

\begin{figure}[!ht]
  \begin{center}
       \makebox[\textwidth][c]{\includegraphics[width=1.2\textwidth]{figures/figure4.pdf}}
    \caption{The results of scenarios in which each land use objective is willing to pay 5\% more cost to compromise with the other three objectives. Cropland converted under each compromise scenario is shown in the maps, colour-coded according to which objective is making the 5\% compromise. Large bar charts, colour-coded to the maps, display the absolute value of costs resulting from each compromise scenario, together with the maximum, minimum, and median costs across all weighting permutations (grey horizontal lines in bar charts). Small bar charts show the percentage of overlap (top) and adjacency (bottom; nearest neighbor distance) between the map directly above it, and each of the other three maps, with the colour denoting which map is being compared.}
    \label{default}
  \end{center}
\end{figure}

We also analyzed an ``equal compromise''  scenario in which each objective receives equal weight, as shown in Figure 5.  The bar charts show how this scenario, represented in purple, compares with each of the 5\% compromise scenarios shown in Figure 4, plus the best and worst case options among all weighting scenarios. Relative to the worst case, equal compromise avoids 80\% or more of the cost for carbon, biodiversity, and transportation, and 41\% of the cost for land area. The equal weighting scenario compares favorably to the $\sim$95\% savings achieved by carbon, biodiversity, and transportation costs within their individual cost categories, and to the 58\% savings by the yield objective. 

As a general rule, the areas selected for conversion under equal compromise lie along existing major transportation corridors, and generally near areas of existing cropland.

 \begin{figure}[!ht]
  \begin{center}
       \makebox[\textwidth][c]{\includegraphics[width=1.2\textwidth]{figures/figure5.pdf}}
    \caption{Areas converted to new maize and soybean croplands and their associated costs when when all objectives (yield maximization, and carbon, biodiversity, and transport cost minimization) compromise equally (i.e. receive 25\% weight). Cropland converted under this scenario is denoted by purple areas in the map, with its associated cost impacts shown alongside those from the 5\% compromise scenarios (Figure 4), together with the maximum, minimum, and median costs across all weighting permutations (grey horizontal lines in bar charts).}
    \label{default}
  \end{center}
\end{figure}

All of these scenarios also suggest the value of deliberate and optimized land use planning.  Zambia has designated a large number of farm blocks, and added nine more blocks with roughly one million hectares as recently as 2005 \citep{anonymous_farm_2005}. Although these areas have as a whole attracted relatively little development, there is continued interest in developing them \citep{anonymous_undeveloped_2015}. But there is less than 2\% overlap between designated farm blocks and either the yield maximization scenario and the equal weighting scenario, and very little adjacency (see supplementary results). 

\section*{\large Discussion}

Our results for Zambia suggest at least the potential to plan agricultural expansion such that it limits carbon and biodiversity costs with limited sacrifice of yield potential.  In fact, if agricultural production goals alone are best represented by a combination of agronomic potential and existing road access, then solutions that provide a good balance of these objectives alone also correlate strongly with solutions that limit carbon and biodiversity impacts.  

This finding demonstrates the potential benefits of targeting land uses based on their ability to deliver a particular benefit for the lowest possible cost. The value of such an efficiency-based approach, which we apply here to agricultural land use, is in line with findings from the field of conservation planning, where it has been shown that conservation outcomes can be achieved for substantially lower cost when both the economic and ecological value of land is considered \citep[e.g.][]{naidoo_integrating_2006, polasky_where_2008}. 

Another result, if more obvious, is the potential benefits of increasing yields on existing cropland if used to spare land. Actions that boost yields on existing cropland also have the potential to have feedback effects that lead to more local expansion, for example, by boosting the competitiveness and therefore quantity of exports \cite{ewers_increases_2009,angelsen_policies_2010}. Our analysis ignores those potential feedback effects because all options analyzed meet the same production targets, but shows how yield gains could be used to limit environmental impacts. 

The fact that the most advantageous compromises occur along existing transportation corridors also appears to have several  advantages. Road construction has been a primary driver of the location of agricultural expansion \citep{laurance_impacts_2009}. Zambia is already engaged in a program to upgrade existing major roads, but there is some evidence in other parts of Africa that agriculture responds even more to improvements in the quality of smaller, rural roads than to improvements in larger paved roadways \citep{bourguignon_rural_2008,blimpo_public_2013}. Improving rural roads would be one way of focusing Zambia's agriculture development on existing road networks.  Because agricultural development tends to build on the private and public infrastructure that accompanies initial development, the focus on existing areas would also seem likely to be a more robust, strategy for avoiding carbon and biodiversity loss over the long-term \cite{laurance_estimating_2015}. 

Beyond road infrastructure, our results also indicate the importance of assessing current agricultural development plans because of the low overlap between Zambia's agricultural development blocks and the areas selected for any objective. 

We believe this type of the model would be particularly useful for planning agricultural development in any politically heterogeneous environment. Optimization models that require ex ante specification of preferences in mathematical terms work less well for decision-making that must weigh very different objectives in unclear ways.  Our model makes possible an iterative optimization process that allows people to realize their preferences in the face of real information about options.  It is particularly suitable to decisions that will inevitably have strong political elements, and to consensus-building among stakeholders with different preferences.  

For this kind of tool to be truly legitimate and useful on a practical level, we believe several elements are required. First, it should be further developed and applied in an iterative way with the government and other stakeholders so that the costs and benefits of different weighting decisions are fully explored. We are developing plans to work with governments, various stakeholders, and international collaborators to further develop and use this model as an operational decision-support tool in Zambia and other countries in the region.

Second, development planning must reflect and incorporate other real world factors beyond those included in our model. For example, large, multi-lane roads can lead to high rates of wildlife mortality, and serve as effective barriers to migration for large mammals, regardless of other land uses \citep{holdo_predicted_2011}.  Although our ``equal weight'' scenario identifies some target areas in the heavily wooded northeastern part of the country, these concerns would be one reason to explore whether areas closer to existing cropland concentrations in the south might be reasonable substitutes. Such consolidation might also make it easier to support agricultural development with lower attendant infrastructure development costs.  And this biodiversity concern is only one example of other issues, many purely practical, which should, and inevitably will, influence where agriculture develops.   
 
Third, data quality is critical. This kind of model focuses on particular hectares, not broad averages, so data errors have little or no chance of averaging out through aggregation. Predicted potential yields and most of our other scores can be highly sensitive to errors in the inputs used to calculate them. These facts mean that long-term use of the model should be accompanied by steady data improvement.  It also means that the model should mostly serve as a guide to likely values. Before governments make any important and potentially irreversible decisions, there should be site-specific ground-truthing of soil properties and yield potential, as well as confirmation of biodiversity and carbon characteristics. 
 
Fourth, some elements of the model should also be further developed. Analysis of agricultural potential could incorporate more economic factors, differentiate between smallholder and commercial farming practices, and incorporate irrigation potential \citep[e.g.][]{nijbroek_regional_2016}. Additional information on land ownership and title would allow a finer delineation of potential agricultural land, while the carbon estimates would be improved by root-shoot specific to each vegetation type. For biodiversity, defining rarity/threat in relation to national borders may provide a biased view of conservation importance--habitats that are rare in Zambia might be common across the border, while a relatively common habitat in Zambia might be regionally rare. A regional approach to defining this measure might therefore be more appropriate \citep[e.g.][]{van_breugel_environmental_2015}. This index could also be adjusted to factor in additional indicators of biodiversity value, such as Important Bird Areas \citep{butchart_protecting_2012} or Key Biodiversity Areas \citep{eken_key_2004}. 

Analysis of biodiversity impacts could also improved by more direct quantification of biodiversity responses to land use. One potentially promising approach would be to incorporate the biodiversity impact model developed by Newbold et al. \citep{newbold_global_2015}, which estimates changes in species diversity and abundance in response to landcover and land-use intensity. With sufficient data, this approach could also allow comparison of the biodiversity impacts of different agricultural practices, including land-sparing approaches that maintain trees, vegetated field borders, or other habitats within the agricultural landscape \citep{kremen_reframing_2015}. The model can be easily adjusted to simulate varying intensities of agricultural management. 

The model could also be used to evaluate the temporal persistence of tradeoffs in response to key uncertainties, such as the future effects of climate change on agricultural potential. Robustness to such uncertainty should be a key factor in determining new agricultural development areas. Assessment of robustness can be achieved by identifying stable areas of conversion that emerge over many iterations within the bounds of uncertainty, which is a feature of the model that we intend to develop.  

Despite the current limitations of the model and this initial analysis, our results are promising. They suggest that efficiency-based land use planning approaches such as ours, if operationalized and incorporated into decision-making processes, can help to substantially minimize the ecological cost of cropland expansion in sub-Saharan Africa's savannas, the 21st Century's emerging hotspot of agricultural land use change. 

\section*{\large Data accessibility}
The model source and inputs data are available online at https://github.com/PrincetonUniversity/agroEcoTradeoff.

\section*{\large Authors' contributions}
L.E. and T.S. conceived and designed the research, together with K.C. L.E. and M.S. designed and coded the model. L.E. developed the input datasets, and ran and analyzed simulations, together with T.S. D.T. and L.E. developed the crop yield potential surfaces. L.K. and T.K. helped design the biodiversity impact metric. S.S. and M.M. collected and processed spatial datasets used in developing inputs. A.B. processed infrastructure data and helped develop the travel cost surface. J.S., N.C., and E.W. developed the bias-corrected weather data needed to estimate yield potential. L.E. and T.S. wrote the manuscript, with input from K.C., T.K., and L.K. 


\section*{\large Competing interests}
We declare we have no competing interests.

\section*{\large Funding}
This work was supported by funds from: the Norwegian Aid Agency (NORAD) under the Agricultural Synergies Project; the Princeton Environmental Institute Grand Challenges program; the NASA New Investigator Program (NNX15AC64G), and the National Science Foundation (SES-1360463 and SES-1534544).  

\section*{\large Acknowledgements}
We thank Paulo van Breugel for providing the potential vegetation map of East Africa, and Philip Thornton for advice regarding modeling of yield potential.    

\section*{\large Supplementary materials}

%\section*{\large References}
\bibliographystyle{prsb} 
\bibliography{/Users/lestes/Dropbox/publications/fullbib}

\end{document}

\endinput
%%
%% End of file `elsarticle-template-harv.tex'.
