%%%% added from https://groups.google.com/forum/#!topic/latexusersgroup/CDlEjgNnF80

%\documentclass[onecolumn]{rsauthor}    
\documentclass[a4paper]{article}
\usepackage[affil-it]{authblk}    


\usepackage{graphicx}
\usepackage{tabularx}
\usepackage{epstopdf}
\usepackage{amsmath}
\usepackage{amssymb}
\usepackage{amsfonts}
\usepackage{amsthm}
\usepackage{endfloat}
\usepackage[numbers,sort,compress]{natbib}
%\bibpunct{[}{]}{,}{n}{,}{,}  % https://xianblog.wordpress.com/tag/natbib/ (allows natbib with PNAS)
\usepackage{endnotes}
\usepackage{setspace}
\usepackage{verbatim}
\usepackage[left=2.5cm, right=2.5cm, bottom=2cm, top=2cm]{geometry}
\usepackage{times}
\usepackage{helvet}
\usepackage{courier}
%\usepackage{mathtime}
\usepackage{bm}
\usepackage{url}
%\usepackage{babel}
\usepackage{dcolumn}
\usepackage{multirow}
%%%%%%

\usepackage[
  breaklinks=true,
  colorlinks=true,
  linkcolor=blue,anchorcolor=blue,
  citecolor=blue,filecolor=blue,
  menucolor=blue,pagecolor=blue,
  urlcolor=blue]{hyperref}

% Let's add todonotes and comments:
%\usepackage[draft]{todonotes}
%
%% Select what to do with command \comment:  
%% \newcommand{\comment}[1]{}  %comment not showed
%\newcommand{\comment}[1]
%{{[\bfseries \color{blue} #1]}} %comment showed

\usepackage{lineno}
\usepackage{float}
\usepackage[anythingbreaks]{breakurl}

\title{Reconciling agriculture, carbon, and biodiversity in a savanna transformation frontier}
\author[1,2]{Estes, L.D.}
\author[2]{Searchinger, T.}
\author[2]{Spiegel, M.}
\author[2]{Tian, D.}
\author[3]{Sichinga, S.}
\author[3]{Mwale, M.}
\author[4]{Kehoe, L.}
\author[4]{Kuemmerle, T.}
\author[1]{Caylor, K.K.}
\affil[1]{Civil and Environmental Engineering, Princeton University, Princeton, NJ, 08544 USA}
\affil[2]{Woodrow Wilson School, Princeton University, Princeton, NJ, 08544 USA}
\affil[3]{Humboldt University, Berlin}
\affil[4]{Humboldt University, Berlin}
\date{}

\begin{document}
\maketitle
%\begin{frontmatter}

\begin{abstract}
Blah blah
\end{abstract}
%% Text of abstract

%\begin{keyword}
%keywords
%\end{keyword}

%\end{frontmatter}


\linenumbers

%% main text
\section*{\large Introduction}

   Meeting growing food demands while minimizing loss of native ecosystems and their carbon and biodiversity presents one of the world�s great challenges for food and the environment.  Many global sustainability studies and climate mitigation pathways call for eliminating or rapidly phasing out emissions from land use change.  CITATIONS IPCC AR V; UK SCIENCE OFFICE.  Unfortunately, these ambitions do not appear realistically attainable in sub-Saharan Africa due to a population likely to more than double to 2.1 billion by 2050. For example, even with healthy linear yield growth from 2006 to 2050 at 3\% to 5\% per year projected by the UN Food and Agriculture Organization (FAO), the region would likely need to add more than 100 million hectares of cropland by 2050.  Searchinger et al. 2013. Much and probably the great bulk of that new cropland is likely to occur in the ~500 million hectares of savannas and shrublands of the region that receive sufficient rainfall to permit crop production.  Searchinger et al. 2013.  This conversion would release tens of millions of tons of carbon and have large impacts on biodiversity.

[I would add in here something about the fact that this is a major challenge, it is one that needs careful planning to achieve, but that this planning has to be done at the country level as this is where most land use policy decisions are made.  Here we look at Zambia, which is perhaps best representative of the challenge.  [this ties us back to the call we made in the NCC paper for finer-level planning, and also inoculates us from the parochiality problem, as we are looking at a representative country].] 

Zambia presents an acute part of the challenge.  According to the medium estimate of the United Nations, Zambia�s population is likely to grow 265\% from 16.2 million in 2015 to 43 million in 2050.  (United Nations Department of Economic and Social Affairs Population Division.  2015.  World Population Prospects:  The 2015 Revision, DVD Edition).   As per capita calorie food availability in Zambia is also low, hunger indices high (with child stunting rates at  40\%), https://www.wfp.org/stories/10-facts-about-hunger-zambia GET BETTER CITATION) , and the percentage of the population dependent on agriculture at X\%, Zambia has great need to boost its agricultural production.  Zambia has also been expanding its agricultural exports, and has agronomic and situational characteristics that put it in a strong position to expand exports to other parts of the region.[1] 
 
At the same time, two thirds of Zambia�s 75 milloin hectares is dominated by Miombo Woodland considered dense to be categorized as forest by the UN REDD program, or X\%.  (Vinya, R., Syampungani, S., Kasumu, E.C., Monde, C. \& Kasubika, R. (2011). Preliminary Study on the Drivers of Deforestation and Potential for REDD+ in Zambia. A consultancy report prepared for Forestry Department and FAO under the national UN-REDD+ Programme Ministry of Lands \& Natural Resources. Lusaka, Zambia).  Using this measure of forest, Zambia�s deforestation rate has been estimated at 250,000 to 300,000 hectares per year, among the highest rates of deforestation of any country in the world. Ibid. Fifty percent of these X million hectares have carbon stocks in excess of X tC/h, and an additional X\% have carbon stocks in excess of X.   
 
The biodiversity of these woody savannas is also high.  Overall, the wetter woodland-savannas and shrublands of Africa have a similar average richness for birds and mammals as the world�s wetter tropical forests, and 75\% of these habitats have higher overall vertebrate diversity than 75\% of the rest of the world.   (Searchiner et al. 2015).   Most of Zambia is covered by what the World Wildlife Fund has categorized as Central Zambian Miombo Woodland.  It contains mature miombo woodland trees typically 15 to 20 meters tall, around a quarter of which are evergreen, with a broadleaf shrub and grass understory beneath.[2]  It is also interspersed with a large quantity of wetlands, up to 30\% of many areas.   Its large expanse of habitat make it a center of large mammals, including elephant, black rhino and African buffalo.  Although endemism is low, birdlife is rich with X species, and reptile and amphibian endemism is high, with 19 endemic reptile species 13 endemic amphibians.  LYNDON, CAN WE ADD SOME NUMBERS OF PLANT SPECIES, AND VERTEBRATE SPECIES.   Zambia contains eight other ecoregions according to WWF classifications.[3]   Its extensive wetlands are particular centers of bird and animal life and under extensive pressures.  The value of this habitat is based in part on its remaining relative abundance, which makes it possible for wide-ranging mammals, including many predator species, to survive. 
 
These multiple trends imply challenging trade-offs between agricultural development and environmental concerns, and raise questions about what modeling approaches might best enable a society of multiple stakeholders to find optimal solutions.  There is a long history of conservation planning that attempts to target conservation areas or objectives while minimizing economic costs or fitting a specific land use budgets for conservation \cite[e.g.][]{koh_spatially_2010, naidoo_integrating_2006,margules_systematic_2000}. Koh and Ghazoul \citep{koh_spatially_2010} provide one example of an effort to examine trade-offs among production, carbon and biodiversity for oil palm development in Indonesia by giving each land use priority equal weighting.  

Focusing on the same categories of objectives, we here present a new model a new model in which varying weights can be placed on the different objectives in an iterative approach to allow decisionmakers and stakeholders to find optimal allocations of new cropland.  This kind of approach can enable different stakeholders to find compromises between competing land use interests while still meeting production targets. We apply this model to Zambia to answer a basic but important question: is it possible to achieve a reasonable balance of different interests, and to reduce the carbon and biodiversity costs of expansion with limited impact on agricultural potential? This question is particularly important for tropical savanna shrubland, and grassland biomes, which are commonly assumed to provide a low environmental cost reserve for agricultural development \cite{searchinger_high_2015}. 

\section*{\large Materials and Methods}
The model we developed, called \emph{AgroEcoTradeoff}, is designed to find the best land use configuration to satisfy the production targets for multiple crops while minimizing four costs: the total land area required, which is inversely related to productivity; the potential losses of carbon and biodiversity; and transportation costs, which are closely correlated with agricultural production costs (ref).  The model takes the basic form:

\begin{equation}
B_{ijk} = \frac{J_{jk}}{P_{ij}} \quad i \in m, \ j = 1, \ldots,n, \ k = 1, \ldots,q 
\end{equation}

Where J$_{k}$ is one of the four costs to be minimized, within a landscape composed of $n$ grid cells containing values for each of the $q$ cost objectives and potential productivity values $P_i$, for each of $m$ crops. $B_{ijk}$ therefore is an efficiency that describes the ecological cost per unit of production, which is normalized so that values are between 0 and 1, so that the different objectives can be compared in common units on a common scale. The normalized values are subtracted from 1 to equate high values with high probability of conversion. Each of these probabilities is then multiplied by a weight (range 0-1, constrained to sum to 1 across all four weights), and the four resulting products are summed, producing a scalarized single objective that represents the landscape conversion probability for each individual crop in each grid cell. An objective function seeks to maximize the sum of these conversion probabilities, based on the requirement that the sum of the potential production of the cells converted for each crop must meet that crop's production target. A second constraint restricts each cell so that it is converted to no more than one crop, as most crops are grown during the same season, and producing multiple crops on the same land would introduce further complexity in setting assumptions about the share of each crop allocated to shared pixels. We therefore formulated a greedy approximation algorithm, based on solving the knapsack problem  \citep{dantzig_discrete-variable_1957} and an algorithm for solving the generalized assignment problem \citep{cohen_efficient_2006}, which allocates each crop to its highest value cells. 

This formulation allows quantitatively optimal solutions to be found across a wide range of weight combinations, which is valuable because it is rare to find a single solution that simultaneously obtains the lowest cost for each objective. By finding the optimal solutions across a range of weights, the model can identify the Pareto front, or the set of solutions where the cost in one objective cannot be further minimized without increasing the cost in another. This set then allows model users to choose the most appealing combination of weights based on their land use preferences, and their own assessments of what are and are not acceptable costs within each of the four competing objectives.  

This weighting system thus has two basic components.  One reflects the pure value judgment for one output versus another, e.g., how much more a person cares about yield over carbon, or carbon over biodiversity.  The other component expresses each output in quantitative terms so that it reasonably expresses preferences for that output.  For example, units for the carbon, biodiversity and yield outputs are scaled so that someone seeking to weight yield at twice the value each of carbon and biodiversity can do so and have the model find an outcome that reasonably reflects this preference. This approach is necessary to support public decision-making processes, in which different individuals and stakeholders might assign different values to different interests, and where compromise solutions must be sought. 

\subsection*{Model inputs} 
For conducting our analysis of potential land use tradeoffs in Zambia, we developed the following spatial datasets at a resolution of 1 km$^2$.    

\subsubsection*{\emph{Agricultural potential and production scenarios}}
There are many ways of evaluating agricultural potential factoring in agronomic and economic characteristics.  Here we focused on two crops, maize and soybeans, because of their existing and potential significance for Zambia�s agricultural sector, and because they are considered to be the two dominant commodities that will be grown in Africa's broader savanna regions \citep{searchinger_high_2015}.  Maize is the largest crop by area and consumption in Zambia \citep{fao_fao_2012}, while soybeans are expanding rapidly as a commodity crop to meet rapidly growing regional and global demands for animal feed and direct human consumption \citep{gasparri_emerging_2015}. 
 
We mapped potential yields for each crop using a three-step approach. In the first step, we used the decision support system for agrotechnology transfer \cite[DSSAT, ][]{jones_modelling_2003} to simulate maize and soybean yields at the sites of 40 weather station distributed across Zambia, which represent the points where a 31-year, gridded daily meteorological dataset has received the greatest amount of bias-correction and infilling by weather observations \citep{sheffield_development_2006,estes_changing_2014}. We used soil profiles corresponding to those locations drawn from the WISE v1.1 gridded soil profile database \citep{harvestchoice_converting_2010}, and simulated for each location maize yields over all 31 years under commercial input practices for three-four cultivars representative of short and medium season length according to the Zambian crop calendar \cite{grassini_how_2015}.  This step allowed us to characterize crop responses to Zambia�s inherent climatic variation, and to improved management practices, which was otherwise impossible given the coarse resolution of available observed yield data and the fact that much of Zambia�s current production occurs on low input, smallholder farms (cite). In the second step, we used an empirical model to map the DSSAT-modeled yields onto a 1 km grid, taking advantage of a finer resolution soil map that provides a common frame for estimating both yield potential and soil carbon stocks (see below), but has insufficient detail in several parameters (particularly effective rooting depth) required by DSSAT.  We used a generalized additive model \citep{wood_mgcv:_2001} to predict the values of the DSSAT-simulated yields, using best-fitting subsets of the following weather and soil variables: growing season precipitation; growing degree days; mean growing season shortwave solar radiation; percent organic carbon; pH; percent clay content. The soil predictors represented mean values in the top 1 m of soil, and were extracted from a new 1 km global database of soil properties \cite{hengl_soilgrids1km_2014}.  These first two steps were necessary to 1) capture crop responses to Zambia's climatic variability and potential management practices, and 2) to base potential yields on the same dataset used to estimate carbon costs (see below). This resulted in a map of potential yield under relatively high inputs and good management, and assuming no losses due to other limitations such as diseases. These yield values therefore overestimated what farmers are truly likely to achieve. To correct this, we therefore rescaled the yield maps to match the mean yields (4.5 t/ha for maize and 3.6 t/ha for soybean) that are projected to be achieved in Zambia by 2050, using FAO-projections to derive an annual growth factor, which we then applied to the 2009-2014 reported mean yields for each crop. Overall, the distribution of yields is most important to targeting land use, but the average yield determines the total quantity of land needed to meet production targets. 

As a second measure related to agricultural potential, we estimated the travel time to the nearest city of 10,000 people or more (check this), which serves as a proxy for production costs, such as fertilizer prices and transport costs (citation), and is strongly correlated with the suitability of land for farming (cites from ConBio paper).  We used a cost-distance analysis drawing on spatial data for roads, rivers and water bodies, urban centers inside and neighboring Zambia, and national boundaries to calculate travel time in hours. Further details on the cost distance and yield estimation methods can be found in the online electronic supplement. 
 
\subsubsection*{\emph{Carbon}} 
To estimate potential carbon costs from cropland conversions, we developed maps of vegetative and soil carbon stocks for Zambia. Vegetative carbon was calculated from a map of above-ground biomass developed by Baccini et al. \citep{baccini_estimated_2012}, from which we estimated below-ground biomass using savanna and miombo woodland-specific root-shoot ratios \cite{mokany_critical_2006,mugasha_allometric_2013}, and then converted the sum to carbon using a ratio of 0.47 (IPCC 2006 cite). We used the same soil carbon dataset used for crop yield prediction together with an accompanying map of bulk density to calculate soil carbon stocks in the top 1 m. The potential carbon loss due to agricultural conversion was calculated as 100\% of vegetative carbon and 25\% of soil carbon \cite{searchinger_high_2015}. 
 
\subsubsection*{\emph{Biodiversity}}
There are many ways of evaluating biodiversity value and potential impacts, including species richness of different taxa in absolute terms or based on level of threat or endemism, and including factors designed to address the significance of contiguity or the impact of different types of development \citep[e.g.][]{newbold_global_2015,searchinger_high_2015, gasparri_emerging_2015,koh_spatially_2010}. Such measures should reflect both conservation priorities and the quality of available data.  For a fine-scaled, country-level analysis such as this one, the species range map data that form the basis for many impact assessments (Kehoe et al, 2015) have an effective resolution that is too coarse (Laura's cite). We therefore developed a composite measure of biodiversity value based on several datasets. First, we used the potential vegetation maps of East Africa \citep[][]{van_breugel_potential_2011}, a 2010 landcover dataset for Zambia (SERVIR cite), and a shapefile of Zambian protected areas to calculate how much of Zambia's 20 unique vegetation types remains after conversion to croplands and settlements, and what proportion of that remainder falls within national parks. From this we calculated a hybrid measure of rarity and threat, wherein we assigned a weight to each vegetation type based on its relative remaining area in proportion to Zambia's extent, log-transformed to account for a highly skewed distribution, and then re-weighted that weight by each vegetation types protected fraction, following \citep{van_breugel_environmental_2015}. We then used the remaining vegetation map to calculate a second measure, the proportion of untransformed vegetation within a 11X11 km neighborhood centered on each 1 km$^2$, providing a measure of vegetation intactness. By adding this measure to the first index and normalizing, we created a unitless biodiversity score (range 0-1) that gives equal weight to how rare or threatened a habitat is, and how undisturbed it is. [add rationale here]

As a final modification, we masked out the areas of existing national parks and game management areas, on the assumption that they would not be used for expanding farmland, but retained a third protected category, forest reserves, as these tend to be far less stringently protected, having had 8.3\% of their area converted to cropland or settlements according to the landcover data. We assigned these areas a high biodiversity score of 0.917 (1 - the already converted proportion), indicating the value of keeping such areas protected, but recognizing that they have some chance of being converted, which can be estimated from their current degree of transformation. 

\subsubsection*{\emph{Potential farmland and production targets}}
Our tradeoff analysis is concerned with cropland expansion into areas that are currently not farmed but could be. To identify such areas, we calculated the proportion of currently cropped and settled grid cells within each 1 km$^2$ grid cell, and used a 30 m digital elevation model to calculate slope percent over the entire country, and calculated the proportion of cells greater than 20\%. Slopes above this percent are considered unsuitable for farming (USDA cite), and slope-based measures are strong predictors of cropland presence or absence \citep{estes_using_2014}. Adding the two resulting maps together, subtracting their value from 1, and removing protected areas provided the fractional potential cropland in each cell. In addition to defining cropland availability, this map also provided the area term necessary for calculating production from potential yield. 

To develop the production targets used by the model, we calculated the rate of change in annual production of maize for Zambia between 2000-2014, collected from the FAOStat database. For soybean we did the same, but used the mean production trend across the 9 southern African countries examined by Gasparri et al \citep{gasparri_emerging_2015}, given that soybean production is very underdeveloped in Zambia compared to countries such as South Africa that are currently experiencing rapid growth. Extrapolating linearly from 2015 until 2050, maize production will triple and soybean production will increase 9-fold. We used these linear extrapolations as the basis for our targets, adding an additional unit to each factor to account for the fact that Zambia, which has one of the highest agricultural potentials in sub-Saharan Africa, is likely to be subject to larger demands for exports to many of its neighbors, thus our final targets were a four-fold increase for maize (from 2009-2014 average production of 2,912,000 tons) and a 10-fold increase for soybean (from 203,700 tons).  

Given that current Zambian maize and soybean yields are well below their current potentials (Waddington et al, 2010), it is an open question how much of future production needs will be satisfied by boosting production on existing croplands. However, it seems likely that some portion of this target will come from current land, given recent Zambian yield growth trends (FAOStat).  In our production scenario, we therefore assumed that an amount equal to a quarter of the gap (redo) between current and projected yields would be met on current farmland before significant expansion occurs.  

\subsection*{Simulations}
Using the selected production targets, we ran the model across all combinations of weights for the four parameters, varying the weights in 5\% increments. This allowed us to evaluate the impact of different land use preferences on the location of conversion impacts, the range and magnitude of potential conversion impacts, including for each objective, and to test how sensitive costs in each objective were to changes in weights.  

\section*{\large Results}
The first order determinant of land use conversion costs is unrelated to any weights applied to the four objectives, as it is based on how much of future production needs will be met on existing croplands, and therefore how much non-agricultural land will be spared from conversion. In our main scenario, we assumed that the gap between current and our projected future potential yields would be fully closed, but to illustrate the impact of this assumption on land conversion costs, we also ran the model (with equal weights across all objectives) for cases when 50\% and 150\% of the gap was closed. In the former case, the total land area required to meet the target increased by 18\% (14,634 to 17,202 km$^2$), average travel time increased 3\% (1.96 to 2.01 hours), carbon losses increased by 19\% (597 to 711 mt), and the average biodiversity index of converted areas increased by 2\% (0.297 to 0.303). In the latter case, the impacts were reversed, with impacts declining by almost exactly the same percentage. 

To identify the minimum possible costs for each individual objective, and the degree to which their areas of new cropland correspond, we examined the simulations were 100\% of the weight was given to each objective. The yield maximizing objective (100\% weight to yield) describes both a cost in terms of the land area required, and a benefit that is based on the productivity gained from converting that land. Scenarios that involve the lowest land use costs also provide farmers the highest potential yields, and therefore have economic advantages.  An examination of overlaps shows two interesting results. First, converted areas that would maximize yield have little correspondence with the areas that minimize carbon costs, biodiversity impacts or travel time (Figure 1), as measured along two dimensions: actual shared area, as measured by the percentage overlap between the two maps, and their adjacency to one another, calculated as the average distance between each pixel of one map and its nearest neighbor in the other. Second, the areas selected for minimizing carbon, biodiversity, and transportation costs have high adjacency, albeit little overlap. All three criteria tend to identify lands along existing road networks and population and agricultural centers. The reason for this correlation is that carbon stocks and biodiversity values are usually lowest in areas of high human density, because of high levels of charcoal and wood harvesting (cite) and fragmentation caused by settlements, existing croplands, and road networks, which exist primarily to connect agricultural and urban centers. 

\begin{figure}[!ht]
  \begin{center}
       \makebox[\textwidth][c]{\includegraphics[width=1.2\textwidth]{figures/single_priorities2.png}}
    %\includegraphics[scale=0.5]{fig2.eps}
    \caption{Areas converted to new maize and soybean croplands and their associated costs when each individual objective (yield maximization, and carbon cost, biodiversity cost, and transport time minimization) is given 100\% weight.  The impact categories are color-coded to the specific conversion maps.}
    \label{default}
  \end{center}
\end{figure}
 
One of the goals of this analysis is to determine whether opportunities exist to achieve a reasonable accommodation between different objectives, and here the results are also interesting. While the best case scenario for maximizing yield would convert 13,668 km$^2$, the worst case scenario (which would minimize carbon costs) would convert 15,302 km$^2$, only an 12\% increase, a range in outcomes that is substantially lower than range in potential crop yields (3.7-6.1 t ha$^{-1}$ for maize and 0.8 to 5.1 ha$^{-1}$ for soybean). This result reflects our model's use, and the value, of an efficiency approach; because yield is in the denominator of each objective (e.g. carbon loss per ton of yield), land with better yield potential tends to be selected--even when the yield maximization priority is given no weight.
 
In contrast to the limited range in total converted area, the differences in travel times range from an average of 0.8 hours for each hectare to 6.7 hours, an increase of 710\%.  Carbon costs increased by 156\% (462 to 1184 Mt), while the worst case for biodiversity was 93\% higher, with the mean biodiversity index of converted areas increasing from 0.26 to 0.5.  These differences suggest that there is high potential to reduce transportation, carbon and biodiversity costs for relatively sacrifice of yield potential. 

This potential is illustrated in Figure 2, which shows how much cost can be avoided within different objective by transferring increasing amounts of weight to them from the yield maximizing objective. Avoided cost is the percent difference between the cost paid under a given weight combination and the most expensive case. For carbon, biodiversity, and travel time, the maximum cost occurs when yield is given 100\% weight, which is where the yield objective has the lowest possible cost, as measured by the percent of land saved relative to the largest area converted (because the cropland that is selected is the highest yielding). However, by sequentially adding weights in 5\% increments to both biodiversity and carbon conservation objectives, large reductions in these costs as well as travel time are rapidly realized (because of the correlations between these three measures), but for little cost to the yield objective. For example, adding just 15\% weight each to carbon and biodiversity increases the average area of new lands (i.e. the average yield is decreased) by just 2.7\% (avoided area costs drop to 8\% from 10.7\%), in exchange for avoided costs of 33\% for biodiversity, 47\% for carbon, and 43\% for travel time, which respectively represent 75\%, 78\%, and 49\% of the maximum avoided cost for each objective.
 
\begin{figure}[!ht]
  \begin{center}
       \makebox[\textwidth][c]{\includegraphics[width=0.8\textwidth]{figures/weight_impact3.png}}
    %\includegraphics[scale=0.5]{fig2.eps}
    \caption{The impact of transferring weights from the yield maximization objective to the carbon and biodiversity objectives in 5\% increments, measured in terms of the percent cost that is avoided in each objective relative to the highest cost paid across all weight combinations.}
    \label{default}
  \end{center}
\end{figure}
 
%These substantial cost savings suggest that a user  achieved for relatively little transfer of weight between objectives indicates that a user primarily concerned with acquiring the highest-yielding land for agriculture would have to make relatively little sacrifice to give enormous gains to conservation interests, or to substantially reduce the potential costs of their farming operation. value of compromise in terms that users might be more likely 

Transferring relatively little weight between objectives can therefore result in large savings. However, people involved in making land use decisions are more likely to evaluate actual cost units with respect to how much more they would be willing to pay to compromise with other interests. We therefore evaluated four further scenarios, in which a hypothetical land user with interests in a single objective willingly pays up to 5\% more than the lowest cost case (i.e. the 100\% weight outcome) in order to compromise with other interests. The yield maximizing user is thus willing to ``pay'' for an extra 68,400 km$^2$ of new cropland, the carbon and biodiversity conservationists would respectively sacrifice an additional 2,312,000 tons of carbon and 0.013 biodiversity index units, and the production cost-minded agricultural developer would pay just 4.1 minutes of travel time. Given these sacrifices, we identified how much more the other three interests would have to pay relative to their lowest cost (100\% weight) scenarios, treating each of the three equally, i.e. the amount each has to pay extra is the same percentage of their lowest cost case. 

The impacts of this assessment show that in for all four cases, the carbon and biodiversity, and travel time objectives typically avoid more than 50\% of potential cost when any of the other objectives is willing to pay 5\% more cost (Figure 3), while the travel time objective avoids slightly less than 50\% of potential cost. Because of the small range in converted areas, a willingness to pay 5\% by the yield objective translates to just 50\% cost avoidance, while the 5\% compromise by carbon and biodiversity results in nearly the maximum potential cost in land area. The same compromise by the cost objective results in a bit less than 50\% cost avoidance by the yield objective. 

The location of the conversions associated with each of these compromises show substantial convergence along roads and population and agricultural centers, with overlap between each of the resulting cropland allocations exceeding 30\% between the yield and biodiversity and cost compromises, and 25\% between the biodiversity and carbon compromises. The carbon and cost compromises showed the lowest overlaps ($<$10\%) followed by carbon and yield ($\sim$15\%). The adjacency of each map was high, with all compromise allocations having an average nearest neighbor distance of $<$20km.  

\begin{figure}[!ht]
  \begin{center}
       \makebox[\textwidth][c]{\includegraphics[width=1.2\textwidth]{figures/compromise_5pct.png}}
    %\includegraphics[scale=0.5]{fig2.eps}
    \caption{The impact of a willingness to pay 5\% more cost to compromise with other land use interests by each of the individual interests.}
    \label{default}
  \end{center}
\end{figure}

These weighting scenarios that corresponded to these selected compromises are shown in Table 1. The weight on the objective subject to the 5\% compromise is proportional to the size of its potential avoided cost, which is why yield requires a weight of only 40\% to pay 5\% or less of its lowest cost, whereas cost and carbon, which have the largest range of costs, require 85\% weight. 

\begin{table}[h]
\begin{center}
\caption{Weights corresponding to the scenarios in which each objective was a) willing to pay 5\% more than its lowest possible cost, and b) the costs for the other three objectives were minimized equitably.}
\begin{tabular}{ccccc}
\hline
& \multicolumn{4}{c}{Weights}\\
Objective making 5\% compromise & Yield & Carbon & Biodiversity & Travel time \\
\hline\hline
Yield & 40 & 10 & 5 & 45 \\
Carbon & 0 & 85 & 5 & 10 \\
Biodiversity & 0 & 10 & 55 & 35 \\
Travel time & 15 & 0 & 0 & 85 \\
\hline
\end{tabular}
\end{center}
\label{default}
\end{table}

We also compared the degree to which these relatively small compromises by individual objectives differed from a situation in which all compromised equally (Figure 4). Equal compromise avoids over 80\% of the potential cost in the carbon, biodiversity, and travel time objectives, and 41\% of the cost in the yield objective. This is lower than, but compares favorably to, the $\sim$95\% savings achieved by the first three objectives within their individual cost categories, under their smaller levels of compromise, and to the 58\% savings by the yield objective. 

\begin{figure}[!ht]
  \begin{center}
       \makebox[\textwidth][c]{\includegraphics[width=1.2\textwidth]{figures/compromise_25pct.png}}
    \caption{25\% compromise}
    \label{default}
  \end{center}
\end{figure}

Another interesting feature of these results is that there is little overlap between the land allocated by the model under these scenarios and areas that are designated farm development blocks (<2\% overlap compared to the yield maximization and equal compromise scenarios; online electronic supplement). Zambia has a large number of these area, and as recently as 2005 the Ministry of Agriculture designated a further nine farm blocks with roughly one million hectares for assisted agricultural development.  (Zambia Ministry of Agriculture. 2005. Farm Block Development Plan 2005-07).  Although these areas have in fact as a whole attracted little development, they remain ``on the books'', and there are at least some reports of continued intent to develop them.  Undeveloped farm blocks worry Agriculture Minister, Zambia Daily Mall: https://www.daily-mail.co.zm/?p=29505

%\begin{equation}
%  \textrm{I} = \left(\frac{\textrm{A}}{\textrm{A + C}} + \frac{\textrm{D}}{\textrm{B + D}}\right)0.5
%\end{equation}

\section*{\large Discussion}

Our results for Zambia suggest at least the potential to plan agricultural expansion such that it limits carbon and biodiversity costs with limited sacrifice of yield potential and with general consistency with existing transportation corridors. 
 
The fact that this potential mostly tracks existing transportation corridors would also appear to have several  advantages.  Road construction has been a primary driver of the location of agricultural expansion.  Laurance, Gossam \& Laurance 2009.  Zambia is already engaged in a program to improve existing major roads, but there is evidence that agriculture responds in particular to improvements in the quality of smaller, feeder rural roads to these paved roadways.  Focusing on these expansion areas would therefore appear to be one means of  focusing Zambia�s agriculture development.  Because agricultural development tends to build on the private and public infrastructure that accompanies initial development, the focus on existing areas would also seem likely to be a more robust, long-term strategy for directing agricultural development \cite{laurance_estimating_2015}. Kassali, Rabirou, Ayanwale, Adeolu, Idowu, Ezekiel, \& Williams, Stella. �Effect of rural transportation system on agricultural productivity in Oyo State, Nigeria�. 2012. Vol 113. Journal of Agricultural and Rural Development in the Tropics and Subtropics.
 
This model appears to be the right kind of tool for assisting the planning of agricultural development. Optimization models that require ex ante specification of preferences in mathematical terms work less well for decision-making that must weigh very different objectives.  Our model makes possible an iterative optimization process that allows people to realize their preferences in the face of real information about options.  It is particularly suitable to decisions that will inevitably have strong political elements, and to consensus-building among stakeholders  with different preferences.  

For the tool to be truly legitimate and useful on a practical level, we believe several elements are required. First, it must be further developed and applied in an iterative way with the government and other stakeholders.  Optimization decisions have many degrees of freedom, and closing them requires informed decision-making.
Second, development planning must reflect and incorporate other real world factors beyond these simple four factors.  For example, many factors beyond those incorporated into our biodiversity analysis have importance for biodiversity. For example, large, multi-lane roads can lead to high rates of mortality and serve as effective barriers to migration for large mammals and others regardless of other land uses (Dobson modeling paper about Serengeti highway).  Special areas may be entitled to high protection regardless of our criteria, as our ranking system already excludes protected areas from potential development.  Other infrastructure development will also influence the practicalities of agricultural development. 
 
Third, data quality is critical. This kind of tool focuses on particular hectares, not broad averages, so errors in data have little or no chance to ``balance out''.  Crop modeling predictions and most of our other scores are highly sensitive to a range of data parameters. These facts mean that long-term use of the tool should be accompanied by steady data improvement.  It also means that the tool should mostly be used as a guide to likely values. Before governments will make important decisions ``for keeps'', there should be site-specific analyses of soil properties, tests of soil fertility and validation of potential yield predictions, as well as confirmation of biodiversity and carbon characteristics. 
 
We also believe some elements of the model should be further developed although all depend on significant additional data gathering. Yield potential, and existing transportation costs, are only two indicators of the economic and practical prospects of agricultural development.  Different areas of land development will require different quantities of inputs, including labor. An agricultural potential indicator based on a spatially differentiable economic model that better estimates both likely net returns to agriculture and agricultural labor would improve the model.

Because the input use and therefore economics of commercial and small-scale agricultural development greatly differ, it would be useful to develop the model to differentiate their different potentials. The carbon calculations would benefit from site-specific root-shoot ratios. The biodiversity measures would benefit from incorporation of data on endemism, other wildlife use and contiguity needs of important animals.

The model does not now incorporate analysis of areas with significant irrigation potential but should.

\section*{\large Data accessibility}
\section*{\large Authors' contributions}
\section*{\large Competing interests}
\section*{\large Funding}
\section*{\large Acknowledgements}
This work was supported by funds from: the Norwegian Aid Agency (NORAD) under the Agricultural Synergies Project; the Princeton Environmental Institute Grand Challenges program; the NASA New Investigator Program (NNX15AC64G), and the National Science Foundation (SES-1360463).   

\section*{\large Supplementary materials}

%\label{}

%% The Appendices part is started with the command \appendix;
%% appendix sections are then done as normal sections
%% \appendix

%% \section{}
%% \label{}

%% If you have bibdatabase file and want bibtex to generate the
%% bibitems, please use
%%
%\section*{\large References}
\bibliographystyle{prsb} 
\bibliography{/Users/lestes/Dropbox/publications/full}

\end{document}

\endinput
%%
%% End of file `elsarticle-template-harv.tex'.
