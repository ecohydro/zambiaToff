\documentclass[a4paper]{article}
\usepackage[affil-it]{authblk}  
%\newcommand{\citetapos}[1]{\citeauthor{#1}'s \citeyearpar{#1}}  % apostrophe's in cites
\def\linenumberfont{\normalfont\small\sffamily}
\usepackage{graphicx}
\usepackage{epstopdf}
% From link below: next two lines allow caption to fill the full figure box, and caption formats to be changed (e.g. font)
% http://tex.stackexchange.com/questions/107350/caption-below-the-figure-and-aligned-with-left-side-of-figure
% http://ctan.mackichan.com/macros/latex/contrib/caption/caption-eng.pdf
\usepackage[left=2.5cm, right=2.5cm, bottom=2cm, top=2cm]{geometry}
\usepackage{caption}
\captionsetup[figure]{slc=off, font=footnotesize}
\captionsetup[table]{margin={4cm, 4cm}, slc=off, font=footnotesize}
\usepackage[numbers,sort,compress]{natbib}
\usepackage[skip=2pt,font=normalsize]{caption}

\begin{document}
\section*{\Large SI Results}

\section*{\large Allocation of land between crops}

To compare the degree to which the cropland allocation routine of \emph{agroEcoTradeoff} causes land that is best for maize to be allocated to soybean and vice versa, we ran the model for each crop separately (with 100\% weight on yield maximization) and compared the overlaps between the resulting conversion maps (Figure 1). The two crops were allocated to the same areas in $<$1\% of converted pixels. 

\begin{figure}[!ht]
\captionsetup{width=0.95\linewidth, font=small}
       \begin{center}
       \makebox[\textwidth][c]{\includegraphics[width=0.8\textwidth]{figures/ideal_yield3.png}}
        \end{center}
      \caption{The distribution of converted areas when the model is run separately for maize and soybean when 100\% is placed on yield maximization. The map is color-coded according to which crop was allocated to each location (red for maize, blue for soybean, and yellow for the places where both crops were allocated).}
      \label{fig:default}
\end{figure}

\section*{\large Overlaps with farm blocks}

\begin{figure}[!ht]
\captionsetup{width=0.95\linewidth, font=small}
       \begin{center}
       \makebox[\textwidth][c]{\includegraphics[width=1.2\textwidth]{figures/fb_overlaps.png}}
        \end{center}
      \caption{Simulations run under the 100\% weight to yield maximization and the equal weights scenarios (25\% weight to all objectives), showing the areas of overlap with designated farm development blocks. }
      \label{fig:default}
\end{figure}

\clearpage
\section*{\large Weights associated with 5\% cost compromises}

\begin{table}[ht]
\begin{center}
\captionsetup{width=0.85\linewidth, font=small}
\caption{Weights corresponding to the scenarios in which each objective was a) willing to pay 5\% more than its lowest possible cost, and b) the costs for the other three objectives were minimized equitably.}
\begin{tabular}{ccccc}
\hline
& \multicolumn{4}{c}{Weights}\\
Objective making 5\% compromise & Yield & Carbon & Biodiversity & Transport Cost \\
\hline\hline
Yield & 40 & 10 & 5 & 45 \\
Carbon & 15 & 75 & 0 & 10 \\
Biodiversity & 15 & 5 & 50 & 30 \\
Travel time & 5 & 5 & 0 & 90 \\
\hline
\end{tabular}
\end{center}
\label{default}
\end{table}


%\section*{\large References}
\bibliographystyle{prsb} 
\bibliography{/Users/lestes/Dropbox/publications/full}

\end{document}  

